\documentclass{amsart}

\usepackage{etex}
\usepackage[all,cmtip]{xy}
\usepackage{tikz-cd}
\usepackage{alltt}
\usepackage{graphicx}
\usepackage{upgreek}
\usepackage{varwidth}
\usepackage{amsthm}
\usepackage{amsmath}
\usepackage{amssymb}
\usepackage{url}
\usepackage{stmaryrd}
\usepackage{ifpdf}
\usepackage{bussproofs}
\usepackage{proof}
\usepackage{setspace}
\usepackage{polytable}
\usepackage{enumitem}
\usepackage{multicol}

\DeclareMathAlphabet{\mathkw}{OT1}{cmss}{b}{n}

\newenvironment{ptree}
{\varwidth{.9\textwidth}\centering\leavevmode}
{\DisplayProof\endvarwidth}
\newlist{pcases}{enumerate}{1}
\setlist[pcases]{
  label=\arabic*:\protect\thiscase.~,
  ref=\arabic*,
  align=left,
  labelsep=0pt,
  leftmargin=0pt,
  labelwidth=0pt,
  parsep=0pt
}
\newcommand{\case}[1][]{%
  \if\relax\detokenize{#1}\relax
  \def\thiscase{}%
  \else
  \def\thiscase{~#1}%
  \fi
\item
}

\newtheorem{theorem}{Theorem}[section]
\newtheorem{corollary}[theorem]{Corollary}
\newtheorem{lemma}[theorem]{Lemma}
\newtheorem{definition}[theorem]{Definition}
\newtheorem{remark}[theorem]{Remark}

\makeatletter
\newenvironment{proofof}[1][\proofname]{%
  \par\pushQED{\qed}\normalfont%
  \topsep6\p@\@plus6\p@\relax
\trivlist\item[\hskip\labelsep\bfseries#1\@addpunct{.}]%
  \ignorespaces
}{%
  \popQED\endtrivlist\@endpefalse
}
\makeatother

\title{The Rules of Combinatory Categorial Grammar}
\author{Jonathan Sterling}

\begin{document}

\begin{abstract}
  A compact formal presentation of the type theory and calculus behind
  the Combinatory Categorial Grammar. Traditional presentations of the
  combining rules tend to involve duplication of a single rule over both
  direction and modality in an effort to make derivations look as much
  as possible like the surface forms they trace. Here, we take a
  different approach and aim for a very succinct syntactic presentation
  which does not visually resemble the surface form in all cases, but
  which preserves enough information to be \emph{interpreted} into such
  a surface form.
\end{abstract}

\maketitle

\def\ty#1{\ensuremath{\mathsf{#1}}}
\def\con#1{\ensuremath{\mathsf{#1}}}
\def\ident#1{\ensuremath{\operatorname{\mathsf{#1}}}}
\def\induction#1{\mathkw{induction}\ #1}
\def\decide#1{\mathkw{decide}\ #1}



\section{The Syntax}

The type theory of combinatory categorial grammar contains base types
which correspond to DPs, VPs, PPs and so forth, as well as function
types which characterize unsaturated terms (such as determiners, verbs,
prepositions, etc.). Words of these types may be combined into larger
phrases by means of several combining rules, of which function
application ($\cdot$) and composition ($\mathbf{B}$) are the most basic.
Others, such as Curry's substitution operator ($\mathbf{S}$) may also be
considered.

Moreover, these combining rules may be applied in either direction so as
to faithfully represent the linear ordering of terms in the surface
form. But some terms may not admit certain combinators, or certain
directions, or some combination of the two constraints: as a result, we
must decorate function types with both permitted direction as well as a
notion of \emph{modality}, which constrains the applicability of
combining rules within the lexicon.

\subsection{Direction and Modality}
A \emph{direction} is one of the set
$\ty{Dir}\triangleq\{\con\triangleright,\con\triangleleft\}$ where $\con\triangleright$
denotes \emph{forward} and $\con\triangleleft$ denotes \emph{backward}.  Directions may
be reversed:
\[
  \arraycolsep=1.4pt
  \begin{array}{lcl}
    !\,\theta &\Leftarrow &\induction\theta\\
    \quad!\,\con\triangleright &\mapsto &\triangleleft\\
    \quad!\,\con\triangleleft &\mapsto &\triangleright
  \end{array}
\]
A \emph{modality} is one of the set $\ty{Mod}\triangleq\{\con\bullet, \con\diamond,
\con\times, \con\star\}$, where \con\bullet\ denotes function terms
which may only be combined using basic function application,
\con\diamond\ those which may be combined by application and
composition such that its operands be uniform in direction, and
\con\times\ those which may be combined by application and composition
such that its operands may \emph{not} be uniform in direction; this kind
of composition is called \emph{crossed} composition. Finally, \con\star\
represents those terms which may be combined using any of the combining
rules.

\begin{theorem}\label{thm:modality-is-lattice}
  The tuple $\langle\ty{Mod}, \land, \lor, \star, \bullet \rangle$
  is a bounded lattice characterized by the following Hasse diagram:
  \begin{center}
    \begin{tikzcd}
      {} & \bullet\arrow[rightarrow]{dr}\arrow[rightarrow]{dl} & \\
      \diamond\arrow[rightarrow]{dr} & & \times\arrow[rightarrow]{dl} \\
                                     & \star
    \end{tikzcd}
  \end{center}
  where $\land$ and $\lor$ are binary operations representing the join
  and the meet respectively of two modalities as follows:

  \noindent\begin{minipage}{0.5\linewidth}
    \[
      \arraycolsep=1.4pt
      \begin{array}{lcl}
        a \land b &\Leftarrow &\induction a\\
        \quad \bullet\land b &\mapsto &\bullet\\
        \quad \diamond\land b &\Leftarrow &\induction b\\
        \quad\quad \diamond\land\bullet &\mapsto &\bullet\\
        \quad\quad \diamond\land\diamond &\mapsto &\diamond\\
        \quad\quad \diamond\land\times &\mapsto &\bullet\\
        \quad\quad \diamond\land\star &\mapsto &\diamond\\
        \quad \times\land b &\Leftarrow &\induction b\\
        \quad\quad \times\land\bullet &\mapsto &\bullet\\
        \quad\quad \times\land\diamond &\mapsto &\bullet\\
        \quad\quad \times\land\times &\mapsto &\times\\
        \quad\quad \times\land\star &\mapsto &\times\\
        \quad \star\land b &\mapsto &b
      \end{array}
    \]
  \end{minipage}
  \begin{minipage}{0.5\linewidth}
    \[
      \arraycolsep=1.4pt
      \begin{array}{lcl}
        a \lor b &\Leftarrow &\induction a\\
        \quad \bullet\lor b &\mapsto &b\\
        \quad \diamond\lor b &\Leftarrow &\induction b\\
        \quad\quad \diamond\lor\bullet &\mapsto &\diamond\\
        \quad\quad \diamond\lor\diamond &\mapsto &\diamond\\
        \quad\quad \diamond\lor\times &\mapsto &\star\\
        \quad\quad \diamond\lor\star &\mapsto &\star\\
        \quad \times\lor b &\Leftarrow &\induction b\\
        \quad\quad \times\lor\bullet &\mapsto &\times\\
        \quad\quad \times\lor\diamond &\mapsto &\star\\
        \quad\quad \times\lor\times &\mapsto &\times\\
        \quad\quad \times\lor\star &\mapsto &\star\\
        \quad \star\lor b &\mapsto &\star
      \end{array}
    \]
  \end{minipage}
\end{theorem}

\begin{lemma}[Commutativity]\label{lem:modality-commutativity}
  For modalities $a, b$, we have $a\land b = b\land a$ and $a\lor b
  = b\lor a$.
\end{lemma}
\begin{proof}
  By induction on $a$ and $b$.
\end{proof}

\begin{lemma}[Associativity]\label{lem:modality-associativity}
  For modalities $a, b, c$, we have $a\land (b\land c) = (a\land b)
  \land c$ and $a\lor (b\lor c) = (a \lor b) \lor c)$.
\end{lemma}
\begin{proof}
  By induction on $a$, $b$ and $c$.
\end{proof}

\begin{lemma}[Absorption]\label{lem:modality-absorption}
  For modalities $a, b$, we have $a\land (a\lor b) = a$ and $a\lor
  (a\land b) = a$.
\end{lemma}
\begin{proof}
  By induction on $a$ and $b$.
\end{proof}

\begin{lemma}[Idempotence]\label{lem:modality-idempotence}
  For any modality $a$, we have $a\land a = a$ and $a\lor a = a$.
\end{lemma}
\begin{proof}
  By induction on $a$.
\end{proof}

\begin{lemma}[Bounding]\label{lem:modality-bounding}
  For any modality $a$, we have $a\land\star = a$ and $a\lor\bullet = a$.
\end{lemma}
\begin{proof}
  By induction on $a$.
\end{proof}

\begin{proofof}[Proof of Theorem \ref{thm:modality-is-lattice}]
  By Lemmas~\ref{lem:modality-commutativity}--\ref{lem:modality-idempotence},
  modalities form a lattice; moreover, by
  Lemma~\ref{lem:modality-bounding}, they are a bounded lattice.
\end{proofof}

\begin{corollary}
  We have a partial order $\leq$ on modalities as follows:
  \[
    \arraycolsep=1.4pt
    \begin{array}{lcl}
      a\leq b &\Leftarrow &\decide{a\land b = a}\\
      \quad \con{yes}\ p &\mapsto &\top\\
      \quad \con{no}\ p &\mapsto &\bot
    \end{array}
  \]
\end{corollary}

\def\arrty#1|[#2,#3]#4{#1\,|^{#2}_{#3}\,#4}
\def\rarrty#1|[#2]#3{#1\,/_{#2}\,#3}
\def\larrty#1|[#2]#3{#1\,\backslash_#2\,#3}


\subsection{The Syntactic Types}
For a set $B$ of base categories, the syntactic types are the closure
of $B$ under the function arrow, annotated by direction and
modality:
\[
  \begin{ptree}
    \AxiomC{$b : B$}
    \UnaryInfC{$b : \ty{SynType}_B$}
  \end{ptree}
  \begin{ptree}
    \AxiomC{$X, Y : \ty{SynType}_B$}
    \AxiomC{$\theta : \ty{Dir}$}
    \AxiomC{$\mu : \ty{Mod}$}
    \TrinaryInfC{$\arrty X|[\theta,\mu]Y : \ty{SynType}_B$}
  \end{ptree}
\]
Modulo direction and modality, the notation $\arrty X|[\theta,\mu]Y$
corresponds to a function type $Y\to X$ in ordinary type theory.
Moreover, when direction is known, we abbreviate with the
following notations:
\begin{align*}
  \rarrty X|[\mu]Y &\triangleq \arrty X|[\con\triangleright,\mu]Y\\
  \larrty X|[\mu]Y &\triangleq \arrty X|[\con\triangleleft,\mu]Y
\end{align*}


\subsection{The Term Language}
Whereas in previous presentations of the CCG calculus, introduction
rules for terms have been duplicated by direction, we can present
them succinctly as follows.

\begin{definition}
  A \ty{Lexicon} over base types $B$ is a (meta-)type parameterized by the
  syntactic types over $B$.
  \[ \ty{Lexicon}_B \triangleq \ty{SynType}_B\to\ty{Type} \]
\end{definition}

Terms are parameterized by the lexicon they draw from: by this means,
terms from differing lexicons may not be combined.
\[
  \infer{\ty{SynTerm}_L\, X : \ty{Type}}
  {
    L : \ty{Lexicon}_B &
    X : \ty{SynType}_B
  }
\]

An entry in a lexicon $L$ is a term in $\ty{SynTerm}_L$.
\[
  \infer{x : \ty{SynTerm}_L\,X}
  {
    X : \ty{SynType}_B &
    x : L_B\, X
  }
\]

\def\syn#1:#2{#1 \mathbin{{\bf\color{blue}:}} #2}

For the sake of brevity, we will often use a shorthand $x\syn:X$ for the
judgement $x:\ty{SynTerm}_L\,X$. At this point we are prepared to give
the combining rules in their full form; given a set of base types $B$, a
$\ty{Lexicon}_B$~$L$, a direction~$\theta$, and a modality~$\mu$:

\begin{equation}
  \infer{f \cdot^\theta_\mu x \syn:X}
  {
    X, Y : \ty{SynType}_B &
    p : \con\bullet\leq\mu &
    f \syn: \arrty X|[\theta,\mu] Y &
    x \syn: Y
  }\tag{$\mathbf{App}$}
\end{equation}

As you can see, we were able to express the two directional variants of
$\mathbf{App}$ in one rule by abstracting over $\theta$. The composition
rule is more interesting, as it places further constraints on both the
directions and the modalities in order to generate in one stroke four
different rules: forward composition, backward composition, forward crossed
composition, and backward crossed composition. We can capture these
constraints with a notion of \ty{Turn}.

\begin{definition}
  A \ty{Turn} is an operation on directions licensed by
  constraints on modalities. Therefore, a $\ty{Turn}\,
  \theta\,\mu\,\nu\,\rho$ licenses a function in direction $\theta$
  and modality $\mu$ to be composed with a function in direction
  $\rho$ and modality $\nu$.
  \[
    \infer{\ty{Turn}\,\theta\,\mu\,\nu\,\rho : \ty{Type}}
    {
      \theta,\rho : \ty{Dir} &
      \mu, \nu: \ty {Mod}
    }
  \]
  The identity \ty{Turn} $\con\shortparallel$ is restricted to modalities
  of at least the same power as $\con\diamond$; the crossed \ty{Turn}
  $\con\curlywedge$ is restricted to modalities of at least the same
  power as $\con\times$:
  \[
    \infer{\con\shortparallel : \ty{Turn}\,\theta\,\mu\,\nu\,\theta}
    {
      \theta : \ty{Dir} &
      p : \con\diamond\leq\mu &
      q : \con\diamond\leq\nu
    }
    \qquad
    \infer{\con\curlywedge : \ty{Turn}\,\theta\,\mu\,\nu\,(!\,\theta)}
    {
      \theta : \ty{Dir} &
      p : \con\times\leq\mu &
      q : \con\times\leq\nu
    }
  \]
\end{definition}

With this in hand, the rule for composition may be expressed in one
shot. Given directions $\theta,\rho$ and modalities $\mu,\nu$:

\begin{equation}
  \infer{f \mathbin{\bf B}_t^\theta g \syn: \arrty X|[\rho,\mu\land\nu]Z}
  {
    X, Y, Z : \ty{SynType}_B &
    t : \ty{Turn}\,\theta\,\mu\,\nu\,\rho &
    f \syn: \arrty X|[\theta,\mu] Y &
    g \syn: \arrty Y|[\rho,\nu] Z
  }\tag{$\mathbf{Comp}$}
\end{equation}


\newpage
\appendix
\section{Logical Kit}

\begin{definition}
  For a proposition $P$, $\ty{Decision}\ P$ is the type of proofs or
  disproofs of $P$.
  \begin{prooftree}
    \AxiomC{$P : \ty{Prop}$}
    \UnaryInfC{$\ty{Decision}\ P : \ty{Type}$}
  \end{prooftree}
  \[
    \begin{ptree}
      \AxiomC{$P : \ty{Prop}$}
      \AxiomC{$p : P$}
      \BinaryInfC{$\con{yes}\ p : \ty{Decision}\ P$}
    \end{ptree}
    \begin{ptree}
      \AxiomC{$P : \ty{Prop}$}
      \AxiomC{$np : P \to \ty\bot$}
      \BinaryInfC{$\con{no}\ np : \ty{Decision}\ P$}
    \end{ptree}
  \]
\end{definition}

\begin{definition}
  We consider a relation $R$ on $C$ decidable if for all $x,y$ in
  $C$, $\ty{Decision}\ x\,R\,y$.
  \begin{prooftree}
    \AxiomC{$C : \ty{Type}$}
    \AxiomC{$R : C\to C\to\ty{Prop}$}
    \BinaryInfC{$\ty{Decidable}\ R \mapsto \prod_{x:C}\prod_{y:C} \ty{Decision}\ x\,R\,y : \ty{Type}$}
  \end{prooftree}

\end{definition}

\end{document}
