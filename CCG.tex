\documentclass{amsart}

\usepackage{etex}
\usepackage[all,cmtip]{xy}
\usepackage{tikz-cd}
\usepackage{alltt}
\usepackage{graphicx}
\usepackage{upgreek}
\usepackage{varwidth}
\usepackage{amsthm}
\usepackage{amsmath}
\usepackage{amssymb}
\usepackage{url}
\usepackage{stmaryrd}
\usepackage{ifpdf}
\usepackage{proof}
\usepackage{setspace}
\usepackage{polytable}
\usepackage{enumitem}
\usepackage{multicol}

\DeclareMathAlphabet{\mathkw}{OT1}{cmss}{b}{n}

\newtheorem{theorem}{Theorem}[section]
\newtheorem{corollary}[theorem]{Corollary}
\newtheorem{lemma}[theorem]{Lemma}
\newtheorem{definition}[theorem]{Definition}
\newtheorem{remark}[theorem]{Remark}

\makeatletter
\newenvironment{proofof}[1][\proofname]{%
  \par\pushQED{\qed}\normalfont%
  \topsep6\p@\@plus6\p@\relax
\trivlist\item[\hskip\labelsep\bfseries#1\@addpunct{.}]%
  \ignorespaces
}{%
  \popQED\endtrivlist\@endpefalse
}
\makeatother

\title{The Rules of Combinatory Categorial Grammar}
\author{Jonathan Sterling}

\begin{document}

\begin{abstract}
  A compact formal presentation of the type theory and calculus behind
  the Combinatory Categorial Grammar, in its fully lexicalized
  multi-modal extension. Traditional presentations of the
  combining rules tend to involve duplication of a single rule over both
  direction and modality in an effort to make derivations look as much
  as possible like the surface forms they trace. Here, we take a
  different approach and aim for a very succinct syntactic presentation
  which does not visually resemble the surface form in all cases, but
  which preserves enough information to be \emph{interpreted} into such
  a surface form.
\end{abstract}

\maketitle

\def\ty#1{\ensuremath{\mathsf{#1}}}
\def\con#1{\ensuremath{\mathsf{#1}}}
\def\ident#1{\ensuremath{\operatorname{\mathsf{#1}}}}
\def\induction#1{\mathkw{induction}\ #1}
\def\decide#1{\mathkw{decide}\ #1}

\section{The Syntax}

The type theory of combinatory categorial grammar contains base types
which correspond to DPs, VPs, PPs and so forth, as well as function
types which characterize unsaturated terms (such as determiners, verbs,
prepositions, etc.). Words of these types may be combined into larger
phrases by means of several combining rules, of which function
application ($\cdot$) and composition ($\mathbf{B}$) are the most basic.
Others, such as Curry's substitution operator ($\mathbf{S}$) may also be
considered.

Moreover, these combining rules may be applied in either direction so as
to faithfully represent the linear ordering of terms in the surface
form. But some terms may not admit certain combinators, or certain
directions, or some combination of the two constraints: as a result, we
must decorate function types with both permitted direction as well as a
notion of \emph{modality}, which constrains the applicability of
combining rules within the lexicon.

\subsection{Direction and Modality}
A \emph{direction} is one of the set
$\ty{Dir}\triangleq\{\con\triangleright,\con\triangleleft\}$ where $\con\triangleright$
denotes \emph{forward} and $\con\triangleleft$ denotes \emph{backward}.  Directions may
be reversed:
\[
  \arraycolsep=1.4pt
  \begin{array}{lcl}
    !\,\theta &\Leftarrow &\induction\theta\\
    \quad!\,\con\triangleright &\mapsto &\triangleleft\\
    \quad!\,\con\triangleleft &\mapsto &\triangleright
  \end{array}
\]
A \emph{modality} is one of the set $\ty{Mod}\triangleq\{\con\bullet, \con\diamond,
\con\times, \con\star\}$, where \con\bullet\ denotes function terms
which may only be combined using basic function application,
\con\diamond\ those which may be combined by application and any rules which
preserve the uniform order of its operands (this kind of
rule is called \emph{harmonic}), and
\con\times\ those which may be combined by application and any rules
which permute the order of its operands (this kind of rule is called
\emph{crossed}). Finally, \con\star\
represents those terms which may be combined using any of the combining
rules.

\begin{theorem}\label{thm:modality-is-lattice}
  The tuple $\langle\ty{Mod}, \land, \lor, \star, \bullet \rangle$
  is a bounded lattice characterized by the following Hasse diagram:
  \begin{center}
    \begin{tikzcd}
      {} & \bullet\arrow[rightarrow]{dr}\arrow[rightarrow]{dl} & \\
      \diamond\arrow[rightarrow]{dr} & & \times\arrow[rightarrow]{dl} \\
                                     & \star
    \end{tikzcd}
  \end{center}
  where $\land$ and $\lor$ are binary operations representing the join
  and the meet respectively of two modalities as follows:

  \noindent\begin{minipage}{0.5\linewidth}
    \[
      \arraycolsep=1.4pt
      \begin{array}{lcl}
        a \land b &\Leftarrow &\induction a\\
        \quad \bullet\land b &\mapsto &\bullet\\
        \quad \diamond\land b &\Leftarrow &\induction b\\
        \quad\quad \diamond\land\bullet &\mapsto &\bullet\\
        \quad\quad \diamond\land\diamond &\mapsto &\diamond\\
        \quad\quad \diamond\land\times &\mapsto &\bullet\\
        \quad\quad \diamond\land\star &\mapsto &\diamond\\
        \quad \times\land b &\Leftarrow &\induction b\\
        \quad\quad \times\land\bullet &\mapsto &\bullet\\
        \quad\quad \times\land\diamond &\mapsto &\bullet\\
        \quad\quad \times\land\times &\mapsto &\times\\
        \quad\quad \times\land\star &\mapsto &\times\\
        \quad \star\land b &\mapsto &b
      \end{array}
    \]
  \end{minipage}
  \begin{minipage}{0.5\linewidth}
    \[
      \arraycolsep=1.4pt
      \begin{array}{lcl}
        a \lor b &\Leftarrow &\induction a\\
        \quad \bullet\lor b &\mapsto &b\\
        \quad \diamond\lor b &\Leftarrow &\induction b\\
        \quad\quad \diamond\lor\bullet &\mapsto &\diamond\\
        \quad\quad \diamond\lor\diamond &\mapsto &\diamond\\
        \quad\quad \diamond\lor\times &\mapsto &\star\\
        \quad\quad \diamond\lor\star &\mapsto &\star\\
        \quad \times\lor b &\Leftarrow &\induction b\\
        \quad\quad \times\lor\bullet &\mapsto &\times\\
        \quad\quad \times\lor\diamond &\mapsto &\star\\
        \quad\quad \times\lor\times &\mapsto &\times\\
        \quad\quad \times\lor\star &\mapsto &\star\\
        \quad \star\lor b &\mapsto &\star
      \end{array}
    \]
  \end{minipage}
\end{theorem}

\begin{lemma}[Commutativity]\label{lem:modality-commutativity}
  For modalities $a, b$, we have $a\land b = b\land a$ and $a\lor b
  = b\lor a$.
\end{lemma}
\begin{proof}
  By induction on $a$ and $b$.
\end{proof}

\begin{lemma}[Associativity]\label{lem:modality-associativity}
  For modalities $a, b, c$, we have $a\land (b\land c) = (a\land b)
  \land c$ and $a\lor (b\lor c) = (a \lor b) \lor c)$.
\end{lemma}
\begin{proof}
  By induction on $a$, $b$ and $c$.
\end{proof}

\begin{lemma}[Absorption]\label{lem:modality-absorption}
  For modalities $a, b$, we have $a\land (a\lor b) = a$ and $a\lor
  (a\land b) = a$.
\end{lemma}
\begin{proof}
  By induction on $a$ and $b$.
\end{proof}

\begin{lemma}[Idempotence]\label{lem:modality-idempotence}
  For any modality $a$, we have $a\land a = a$ and $a\lor a = a$.
\end{lemma}
\begin{proof}
  By induction on $a$.
\end{proof}

\begin{lemma}[Bounding]\label{lem:modality-bounding}
  For any modality $a$, we have $a\land\star = a$ and $a\lor\bullet = a$.
\end{lemma}
\begin{proof}
  By induction on $a$.
\end{proof}

\begin{proofof}[Proof of Theorem \ref{thm:modality-is-lattice}]
  By Lemmas~\ref{lem:modality-commutativity}--\ref{lem:modality-idempotence},
  modalities form a lattice; moreover, by
  Lemma~\ref{lem:modality-bounding}, they are a bounded lattice.
\end{proofof}

\begin{corollary}\label{cor:modality-porder}
  We have a partial order $\leq$ on modalities as follows:
  \[
    \arraycolsep=1.4pt
    \begin{array}{lcl}
      a\leq b &\Leftarrow &\decide{a\land b = a}\\
      \quad \con{yes}\ p &\mapsto &\top\\
      \quad \con{no}\ p &\mapsto &\bot
    \end{array}
  \]
\end{corollary}

\def\arrty#1|[#2,#3]#4{#1\,|^{#2}_{#3}\,#4}
\def\rarrty#1|[#2]#3{#1\,/_{#2}\,#3}
\def\larrty#1|[#2]#3{#1\,\backslash_#2\,#3}


\subsection{The Syntactic Types}
For a set $B$ of base categories, the syntactic types are the closure
of $B$ under the function arrow, annotated by direction and
modality:
\[
  \infer{b : \ty{SynType}_B}
  {
    b : B
  }
  \qquad
  \infer{\arrty X|[\theta,\mu]Y : \ty{SynType}_B}
  {
    X, Y : \ty{SynType}_B &
    \theta : \ty{Dir} &
    \mu : \ty{Mod}
  }
\]
Modulo direction and modality, the notation $\arrty X|[\theta,\mu]Y$
corresponds to a function type $Y\to X$ in ordinary type theory.
Moreover, when direction is known, we abbreviate with the
following notations:
\begin{align*}
  \rarrty X|[\mu]Y &\triangleq \arrty X|[\con\triangleright,\mu]Y\\
  \larrty X|[\mu]Y &\triangleq \arrty X|[\con\triangleleft,\mu]Y
\end{align*}


\subsection{The Term Language}
Whereas in previous presentations of the CCG calculus, introduction
rules for terms have been duplicated by direction, we can present
them succinctly as follows.

\begin{definition}
  A \ty{Lexicon} over base types $B$ is a (meta-)type parameterized by the
  syntactic types over $B$.
  \[ \ty{Lexicon}_B \triangleq \ty{SynType}_B\to\ty{Type} \]
\end{definition}

Terms are parameterized by the lexicon they draw from: by this means,
terms from differing lexicons may not be combined.
\[
  \infer{\ty{SynTerm}_L\, X : \ty{Type}}
  {
    L : \ty{Lexicon}_B &
    X : \ty{SynType}_B
  }
\]

An entry in a lexicon $L$ is also a term in $\ty{SynTerm}_L$.
\[
  \infer{x : \ty{SynTerm}_L\,X}
  {
    X : \ty{SynType}_B &
    x : L_B\, X
  }
\]

\def\syn#1:#2{#1 \mathbin{{\bf\color{blue}:}} #2}

For the sake of brevity, we will often use a shorthand $x\syn:X$ for the
judgement $x:\ty{SynTerm}_L\,X$. At this point we are prepared to give
the combining rules in their full form; given a set of base types~$B$
and a lexicon~$L$:

\begin{equation}
  \infer{f \cdot^\theta_\mu x \syn:X}
  {
    X, Y : \ty{SynType}_B &
    \theta : \ty{Dir} &
    \mu : \ty{Mod} &
    p : \con\bullet\leq\mu &
    f \syn: \arrty X|[\theta,\mu] Y &
    x \syn: Y
  }\tag{$\mathbf{App}$}
\end{equation}

As you can see, we were able to express the two directional variants of
$\mathbf{App}$ in one rule by abstracting over $\theta$. We could of
course omit the constraint $p$, since by
Lemma~\ref{lem:modality-bounding} and
Corollary~\ref{cor:modality-porder} we have $\con\bullet\leq\mu$ for all
modalities $\mu$.

Naturally, type-raising can also be expressed very simply using our
direction reversal operator:

\begin{equation}
  \infer{\uparrow^\theta_\mu x \syn: \arrty Y|[\theta,\mu](\arrty Y|[!\,\theta, \mu] X)}
  {
    X, Y : \ty{SynType}_B &
    \theta : \ty{Dir} &
    \mu : \ty{Mod} &
    x \syn: X
  }\tag{$\mathbf{TR}$}
\end{equation}

The composition rule is more interesting, as it places further
constraints on both the directions and the modalities in order to
generate in one stroke four different rules: forward composition,
backward composition, forward crossed composition, and backward crossed
composition. We can capture these constraints with a notion of
\ty{Turn}.

\begin{definition}
  A \emph{turn} is an operation on directions licensed by
  constraints on modalities. Therefore, a $\ty{Turn}\,
  \theta\,\mu\,\nu\,\rho$ licenses a function in direction $\theta$
  and modality $\mu$ to be composed with a function in direction
  $\rho$ and modality $\nu$.
  \[
    \infer{\ty{Turn}\,\theta\,\mu\,\nu\,\rho : \ty{Type}}
    {
      \theta,\rho : \ty{Dir} &
      \mu, \nu: \ty {Mod}
    }
  \]
  The identity turn $\con\shortparallel$ is restricted to modalities
  of at least the same power as $\con\diamond$; the crossed turn
  $\con\curlywedge$ is restricted to modalities of at least the same
  power as $\con\times$:
  \[
    \infer{\con\shortparallel : \ty{Turn}\,\theta\,\mu\,\nu\,\theta}
    {
      \theta : \ty{Dir} &
      p : \con\diamond\leq\mu &
      q : \con\diamond\leq\nu
    }
    \qquad
    \infer{\con\curlywedge : \ty{Turn}\,\theta\,\mu\,\nu\,(!\,\theta)}
    {
      \theta : \ty{Dir} &
      p : \con\times\leq\mu &
      q : \con\times\leq\nu
    }
  \]
\end{definition}

With this in hand, the rules for composition may be expressed each in
one shot, accounting for harmonic and crossed variants in either
direction:

\begin{equation}
  \infer{f \mathbin{\bf B}_t^\theta g \syn: \arrty X|[\rho,\mu\lor\nu]Z}
  {
    X, Y, Z : \ty{SynType}_B &
    \theta,\rho : \ty{Dir} &
    \mu,\nu : \ty{Mod} &
    t : \ty{Turn}\,\theta\,\mu\,\nu\,\rho &
    f \syn: \arrty X|[\theta,\mu] Y &
    g \syn: \arrty Y|[\rho,\nu] Z
  }\tag{$\mathbf{Comp}$}
\end{equation}

It turns out that \ty{Turn} also suffices to give a single rule schema
for substitution (a form of combination in which an argument is used
twice):

\begin{equation}
  \infer{f \mathbin{\bf S}_t^\theta g \syn: \arrty X|[\rho,\mu\lor\nu]Z}
  {
    X, Y, Z : \ty{SynType}_B &
    \theta,\rho : \ty{Dir} &
    \mu,\nu : \ty{Mod} &
    t : \ty{Turn}\,\theta\,\mu\,\nu\,\rho &
    f \syn: \arrty {(\arrty X|[\theta,\mu]Y)} |[ \rho,\nu] Z &
    g \syn: \arrty Y|[\rho,\nu]Z
  }\tag{$\mathbf{Subst}$}
\end{equation}

\subsection{Elaboration and Decidability of Type Checking}

A term in our calculus which represents the syntax of ``The dog ate the
shoe'' can be given as follows:

\begin{equation}\label{eq:dog-ate-shoe}
  (\text{ate} \cdot^\triangleright_{\con\bullet} (\text{the} \cdot^\triangleright_{\con\diamond}
\text{shoe})) \cdot^\triangleleft_{\con\bullet} (\text{the} \cdot^\triangleright_{\con\diamond}
\text{dog})
\end{equation}

We can either consider such a notation as having elided the proofs which
satisfy the constraints of the rules, or as generating proof obligations
which must be satisfied in order to construct a typing derivation. In
the latter case, we can then consider this notation to be a
surface-level representation which is elaborated into a fully explicit
one; in this case, this would mean mechanically adding proofs that
$\con\bullet\leq\con\diamond$ and $\con\bullet\leq\con\bullet$ hold.

\begin{theorem}\label{thm:decidable-ccg}
  Type checking the CCG is decidable.
\end{theorem}
\begin{proof}
  The typing rules of the CCG that of the STLC, but indexed by
  constraints on modality and direction. Because \ty{Dir} has decidable
  equality and \ty{Modality} has a decidable order, type checking is
  decidable.
\end{proof}

\subsection{Correspondence to Visual Proofs}

A syntactic proof in traditional CCG is laid out visually as a trace of
the judgements which lead to a particular string being considered
admissible. It may seem that because we have abstracted direction out of
the judgements, we will no longer be able to construct such proofs, but
this is not so. The fact that the order of (meta)-parameters of
judgements no longer corresponds to the order of elements in a string is
inconsequential: we can choose to have our notation vary with the choice
of directional parameters (such as $\theta:\ty{Dir}$).

It is trivial to convert the term in (\ref{eq:dog-ate-shoe}) into a
derivation of the traditional sort by simply making the order of
elements differ according to the directionality of the combining
operator:
\begin{equation}
  \dfrac{
    \dfrac{
      \dfrac{\text{the}}{\rarrty D|[\con\diamond]N}
      \quad
      \dfrac{\text{dog}}{N}
    }{D}\triangleright
    \quad
    \dfrac{
      \dfrac{\text{ate}}{\rarrty {(\larrty S|[\con\bullet]D)}|[\con\bullet]D}
      \quad
      \dfrac{
        \dfrac{\text{the}}{\rarrty D|[\con\diamond]N}
        \quad
        \dfrac{\text{shoe}}{N}
      }{D}\triangleright
    }{\larrty S|[\con\bullet]D}\triangleright
  }{S}\triangleleft
\end{equation}
This is a purely visual conceit: it makes no difference whether we
introduce direction symbolically or spatially.

\end{document}
